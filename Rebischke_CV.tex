%% The MIT License (MIT)
%%
%% Copyright (c) 2015 Daniil Belyakov
%%
%% Permission is hereby granted, free of charge, to any person obtaining a copy
%% of this software and associated documentation files (the "Software"), to deal
%% in the Software without restriction, including without limitation the rights
%% to use, copy, modify, merge, publish, distribute, sublicense, and/or sell
%% copies of the Software, and to permit persons to whom the Software is
%% furnished to do so, subject to the following conditions:
%%
%% The above copyright notice and this permission notice shall be included in all
%% copies or substantial portions of the Software.
%%
%% THE SOFTWARE IS PROVIDED "AS IS", WITHOUT WARRANTY OF ANY KIND, EXPRESS OR
%% IMPLIED, INCLUDING BUT NOT LIMITED TO THE WARRANTIES OF MERCHANTABILITY,
%% FITNESS FOR A PARTICULAR PURPOSE AND NONINFRINGEMENT. IN NO EVENT SHALL THE
%% AUTHORS OR COPYRIGHT HOLDERS BE LIABLE FOR ANY CLAIM, DAMAGES OR OTHER
%% LIABILITY, WHETHER IN AN ACTION OF CONTRACT, TORT OR OTHERWISE, ARISING FROM,
%% OUT OF OR IN CONNECTION WITH THE SOFTWARE OR THE USE OR OTHER DEALINGS IN THE
%% SOFTWARE.

% The font could be set to Windows-specific Calibri by using the 'calibri' option
\documentclass[]{mcdowellcv}

% For mathematical symbols
\usepackage{amsmath}
\usepackage{hyperref}

% Set applicant's personal data for header
\name{Christian Rebischke}
\address{Marie-Hedwig-Straße 13 \linebreak Apartment 321 \linebreak Clausthal-Zellerfeld 38678}
\contacts{+49 151 6190 2666 \linebreak chris@nullday.de \linebreak \url{https://nullday.de}}

\begin{document}

% Print the header
\makeheader

% Print the content
\begin{cvsection}{Employment}
\begin{cvsubsection}{Student Asisstant}{TU Clausthal, Datacenter}{Apr 2016}

\begin{itemize}
\item I created knowledge via building a proof of concept for deploying Virtual Tunnel End Points (VTEPs) with Ansible on Linux machines for EVPN BGP/VXLAN.
\item I reduced toil from multiple hours to an automated system with automating daily email based firewall IPS/IDS alerts via the Forcepoint NGFW Python API.
\item I improved system security and reliability with setting up an OpenVAS vulnerability scanner.
\item I reduced MTTR from one work day to one hour with automating a Freeradius/Radsecproxy/MySQL based AAA infrastructure with Ansible.
\item I increased virtual machine reliability via creating a Proxmox VE cluster consisting of 25 machines.
\item I reduced toil from multiple hours to an automated system with writing a Python client for deploying TLS certificates and private keys on a Forcepoint NGFW for TLS inspection.
\item Additional tasks were monitoring (NSCA, NRPE, SNMP), webserver (Nginx, Apache), network Automation (NAPALM, Ansible) and storage (NFSv4 over Kerberos).
\end{itemize}
\end{cvsubsection}

\begin{cvsubsection}{Student Assistant}{TU Clausthal Inst. of Software Systems Engineering}{Oct 2016 -- Sep 2017}
\bigskip
\begin{itemize}
\item I build a tool chain for exporting Matlab Simulink models into the Functional Mockup Unit (FMU) format.
\item I developed components for a model transformation tool suite in the project \emph{Spectral Analsysis of Software Architecture}
\item I enhanced code quality with establishing the Continuous Code Quality tool Sonarqube.
\item I used the following technologies for accomplishing these tasks: Java, Gradle, Matlab, SVN
\end{itemize}
\end{cvsubsection}

\begin{cvsubsection}{Student Assistant}{TU Clausthal Inst. of Mathematics}{Apr 2014 -- Sep 2017}
\bigskip
\begin{itemize}
\item I increased system reliability with monitoring via the Nagios fork Centreon and using the protocols NRPE, NSCA, SNMP.
\item I reduced toil with building linux packages for Ubuntu and CentOS.
\item I have administrated Linux and Windows machines and gave first level support.
\item Furthermore I have wrote bash scripts, created a NFSv4 server with Kerberos, managed Apache webserver, CUPS printing server, a Firefox sync server for bookmarks and passwords, and a MySQL server.
\end{itemize}
\end{cvsubsection}

\end{cvsection}

\begin{cvsection}{Education}
\begin{cvsubsection}{Clausthal-Zellerfeld, Germany}{Technical University Clausthal}{October 2013}
\begin{itemize}
\item Bachelor of Science in Computer Science.
\end{itemize}
\end{cvsubsection}

\end{cvsection}

\begin{cvsection}{Open Source Contribution}
\begin{cvsubsection}{Arch Linux}{\url{https://archlinux.org}}{January 2015}
\begin{itemize}
\item \textbf{Security Advisories} Verifying known Common Vulnerabilities and Expores (CVEs) in Arch Linux packages.
\item \textbf{Hardening} Improving Security of Arch Linux packages and infrastructure.
\item \textbf{Package Maintaner} Building source code into Arch Linux binary packages for distribution, committing patches and supporting the community.
\end{itemize}
\end{cvsubsection}

\begin{cvsubsection}{Projects}{\url{https://github.com/shibumi}}{}
\begin{itemize}
\item \textbf{Arch Linux Boxes} Building reliable infrastructure for automated monthly Vagrant builds with Ansible and Hashicorp Packer.
\item \textbf{ProcFS} Adding support for CIFS in the Prometheus Node Exporter component ProcFS.
\item \textbf{Fighting Malware} Participation in fighting global botnets and malware: \url{https://www.virusbulletin.com/uploads/pdf/conference_slides/2015/KalnaiHorejsi-VB2015.pdf}.
\end{itemize}
\end{cvsubsection}
\end{cvsection}

\begin{cvsection}{Languages, Technologies and Interests}
\begin{cvsubsection}{}{}{}
\begin{itemize}
\item Python, Golang, Java, C, C++, Bash, German, English
\item I have a huge interest in Site Reliability Engineering (I've read the book \emph{Site Reliability Engineering} and watched various Youtube videos by Stephen Thorne, Seht Vargo, Liz Fong-Jones, JC van Winkel and Nori Heikkinen).
\item I am 70\% Ops, I hope I can improve to a more SRE healthy 50/50 composition.
\item Besides DevOps and SRE I am interested in security (malware, forensics, binary exploitation, hardening servers and services).
\item What I want to learn: I want to learn how to doing splits between embracing risk and providing system reliability with multiple 9s, how to contribute to big Golang projects like Kubernetes or Istio and how to manage datacenters with a combination of Jupiter, Borg, B4 and Colossus.
\end{itemize}
\end{cvsubsection}
\end{cvsection}

\end{document}
