%% The MIT License (MIT)
%%
%% Copyright (c) 2015 Daniil Belyakov
%%
%% Permission is hereby granted, free of charge, to any person obtaining a copy
%% of this software and associated documentation files (the "Software"), to deal
%% in the Software without restriction, including without limitation the rights
%% to use, copy, modify, merge, publish, distribute, sublicense, and/or sell
%% copies of the Software, and to permit persons to whom the Software is
%% furnished to do so, subject to the following conditions:
%%
%% The above copyright notice and this permission notice shall be included in all
%% copies or substantial portions of the Software.
%%
%% THE SOFTWARE IS PROVIDED "AS IS", WITHOUT WARRANTY OF ANY KIND, EXPRESS OR
%% IMPLIED, INCLUDING BUT NOT LIMITED TO THE WARRANTIES OF MERCHANTABILITY,
%% FITNESS FOR A PARTICULAR PURPOSE AND NONINFRINGEMENT. IN NO EVENT SHALL THE
%% AUTHORS OR COPYRIGHT HOLDERS BE LIABLE FOR ANY CLAIM, DAMAGES OR OTHER
%% LIABILITY, WHETHER IN AN ACTION OF CONTRACT, TORT OR OTHERWISE, ARISING FROM,
%% OUT OF OR IN CONNECTION WITH THE SOFTWARE OR THE USE OR OTHER DEALINGS IN THE
%% SOFTWARE.

% The font could be set to Windows-specific Calibri by using the 'calibri' option
\documentclass[]{mcdowellcv}

% For mathematical symbols
\usepackage{amsmath}
\usepackage{hyperref}

% Set applicant's personal data for header
\name{Christian Rebischke}
\address{Marie-Hedwig-Straße 13 \linebreak Apartment 321 \linebreak Clausthal-Zellerfeld 38678}
\contacts{+49 151 6190 2666 \linebreak chris@nullday.de \linebreak \url{https://nullday.de}}

\begin{document}

% Print the header
\makeheader

% Print the content
\begin{cvsection}{Employment}
\begin{cvsubsection}{Student Asisstant}{TU Clausthal, Datacenter}{Apr 2016 -- April 2020}

\begin{itemize}
\item I created knowledge via building a proof of concept for deploying Virtual Tunnel End Points (VTEPs) with Ansible on Linux machines for EVPN BGP/VXLAN.
\item I reduced toil from multiple hours to an automated system with automating daily email based firewall IPS/IDS alerts via the Forcepoint NGFW Python API.
\item I improved system security and reliability with setting up an OpenVAS vulnerability scanner.
\item I reduced MTTR from one work day to one hour with automating a Freeradius/Radsecproxy/MySQL based AAA infrastructure with Ansible.
\item I increased virtual machine reliability via creating and maintaining a Proxmox VE cluster consisting of 25 machines.
\item I reduced toil from multiple hours to an automated system with writing a Python client for deploying TLS certificates and private keys on a Forcepoint NGFW for TLS inspection.
\item My current job also involves evaluating Kubernetes for increasing reliability and introducing micro segmentation via namespace segregation and implementing my bachelor thesis project on-premise with new additions (Traefik as reverse proxy and load balancer for Prometheus and Grafana).
\item Additional tasks were/are monitoring (NSCA, NRPE, SNMP), webserver (Nginx, Apache), network Automation (NAPALM, Ansible), storage (NFSv4 over Kerberos), log and netflow gathering (Elasticsearch, Logstash, Kibana) and Configuration Management (Ansible).
\item I gave a talk about Freeradius and Radsecproxy deployment via Ansible on the DFN-BT (annual German Research Network Meetup): \url{https://www.dfn.de/fileadmin/3Beratung/Betriebstagungen/bt70/BT70_MobileIT_Konfiguration_FreeRADIUS_und_radsecproxy_mit_Ansible_Strauf_Rebischke.pdf}
\end{itemize}
\end{cvsubsection}

\begin{cvsubsection}{Student Assistant}{TU Clausthal Inst. of Software Systems Engineering}{Oct 2016 -- Sep 2017}
\bigskip
\begin{itemize}
\item I build a tool chain for exporting Matlab Simulink models into the Functional Mockup Unit (FMU) format.
\item I developed components for a model transformation tool suite in the project \emph{Spectral Analsysis of Software Architecture}
\item I enhanced code quality with establishing the Continuous Code Quality tool Sonarqube.
\item I used the following technologies for accomplishing these tasks: Java, Gradle, Matlab, SVN
\end{itemize}
\end{cvsubsection}

\begin{cvsubsection}{Student Assistant}{TU Clausthal Inst. of Mathematics}{Apr 2014 -- Sep 2017}
\bigskip
\begin{itemize}
\item I increased system reliability with monitoring via the Nagios fork Centreon and using the protocols NRPE, NSCA, SNMP.
\item I reduced toil with building linux packages for Ubuntu and CentOS.
\item I have administrated Linux and Windows machines and gave first level support.
\item Furthermore I have wrote bash scripts, created a NFSv4 server with Kerberos, managed Apache webserver, CUPS printing server, a Firefox sync server for bookmarks and passwords, and a MySQL server.
\end{itemize}
\end{cvsubsection}

\end{cvsection}

\begin{cvsection}{Education}
\begin{cvsubsection}{B.Sc. Computer Science}{Technical University Clausthal}{Oct 2013 -- May 2019}
\begin{itemize}
\item Seminar paper: Amazon AWS (EC2 virtual Server and EC2 container) (German) \url{https://github.com/shibumi/aws-ec2-project-paper}
\item Seminar paper: Openstack (internal structure and overview) (German) \url{https://github.com/shibumi/openstack-project-paper}
\item Seminar paper: Tor (a short introduction in The Onion Routing) (German) \url{https://github.com/shibumi/Tor-project-paper}
\item Bachelor thesis: Evaluation of a distributed monitoring system for the TU Clausthal Campus (German) \url{https://github.com/shibumi/bachelor-thesis}
\item Bachelor defense: Evaluation of a distributed monitoring system for the TU Clausthal Campus (German) \url{https://github.com/shibumi/bachelor-kolloquium}
\end{itemize}
\end{cvsubsection}
\begin{cvsubsection}{M.Sc. Computer Science}{Technical University Clausthal}{Oct 2018 -- Oct 2020}
Right now I am working on the theoretical approach for micro service identification and characterization for service matching in the research project \textit{Basic technologies and engineering methods for emergent genesis and semantic composition of IoT ecosystems}. The research project will be finished in April 2020.
\end{cvsubsection}

\end{cvsection}
\begin{cvsection}{Open Source Contribution}
\begin{cvsubsection}{Arch Linux}{\url{https://archlinux.org}}{Jan 2015 -- Now}
\begin{itemize}
\item \textbf{Security Advisories} Verifying known Common Vulnerabilities and Expores (CVEs) in Arch Linux packages.
\item \textbf{Hardening} Improving Security of Arch Linux packages and infrastructure.
\item \textbf{Package Maintaner} Building source code into Arch Linux binary packages for distribution, committing patches and supporting the community.
\item \textbf{Release Engineering} Vagrant, qcow2 and Docker image builds for Arch Linux.
\end{itemize}
\end{cvsubsection}

\begin{cvsubsection}{Projects}{\url{https://github.com/shibumi}}{}
Here I list my projects, some of these projects are currently freezed due to a lack of free time besides university, job in the university datacenter and my Arch Linux contribution.
\begin{itemize}
\item \textbf{Arch Linux Boxes} Building reliable infrastructure for automated monthly Vagrant and qcow2 image builds with Ansible and Hashicorp Packer. This project includes a small python script that reduces the toil to manually check for the monthly needed fresh Arch Linux ISO image. \url{https://github.com/archlinux/arch-boxes}
\item \textbf{nullday.de} My personal blog with a 100/100 TLS Rating \url{https://www.ssllabs.com/ssltest/analyze.html?d=nullday.de} and a 130/100 security rating according to \url{https://observatory.mozilla.org/analyze/nullday.de} with a 100/100 Google PageSpeed Insights rating.
\item \textbf{ProcFS} Adding support for CIFS in the Prometheus Node Exporter component ProcFS (This project is \textit{work in progress})
\item \textbf{nspawn.org} A hub for systemd-nspawn container images and bootable GPT machine images available on \url{https://nspawn.org}
\item \textbf{Fighting Malware} Participation in fighting global botnets and malware: \url{https://www.virusbulletin.com/uploads/pdf/conference_slides/2015/KalnaiHorejsi-VB2015.pdf}. I shared via SSH honepot gathered Linux/XOR.DDoS samples with security researchers: \url{https://blog.malwaremustdie.org/2014/09/mmd-0028-2014-fuzzy-reversing-new-china.html}
\item \textbf{SRE-Cheatsheet} I am working on a small Site Reliability Engineering cheat sheet for beginning SREs: \url{https://github.com/shibumi/SRE-cheat-sheet} (This project is \textit{work in progress})
\end{itemize}
\end{cvsubsection}
\end{cvsection}

\begin{cvsection}{Languages, Additional Technologies and Interests}
\begin{cvsubsection}{}{}{}
\begin{itemize}
\item \textbf{Natural Languages} German, English
\item \textbf{Programming Languages} Bash, Python, C++, Golang, C, Java, x86 Assembly (sorted by skill level from left/expert to right/beginner)
\item \textbf{Interests} Site-Reliability Engineering, Devops, Network Infrastructure, Reverse Engineering, Forensics, Penetration Testing, Red Team/Blue Team, Blackbox/Whitebox Testing, Malware, Server Hardening, Network Security.
\end{itemize}
\end{cvsubsection}
\end{cvsection}

\end{document}
