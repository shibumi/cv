%% The MIT License (MIT)
%%
%% Copyright (c) 2015 Daniil Belyakov
%%
%% Permission is hereby granted, free of charge, to any person obtaining a copy
%% of this software and associated documentation files (the "Software"), to deal
%% in the Software without restriction, including without limitation the rights
%% to use, copy, modify, merge, publish, distribute, sublicense, and/or sell
%% copies of the Software, and to permit persons to whom the Software is
%% furnished to do so, subject to the following conditions:
%%
%% The above copyright notice and this permission notice shall be included in all
%% copies or substantial portions of the Software.
%%
%% THE SOFTWARE IS PROVIDED "AS IS", WITHOUT WARRANTY OF ANY KIND, EXPRESS OR
%% IMPLIED, INCLUDING BUT NOT LIMITED TO THE WARRANTIES OF MERCHANTABILITY,
%% FITNESS FOR A PARTICULAR PURPOSE AND NONINFRINGEMENT. IN NO EVENT SHALL THE
%% AUTHORS OR COPYRIGHT HOLDERS BE LIABLE FOR ANY CLAIM, DAMAGES OR OTHER
%% LIABILITY, WHETHER IN AN ACTION OF CONTRACT, TORT OR OTHERWISE, ARISING FROM,
%% OUT OF OR IN CONNECTION WITH THE SOFTWARE OR THE USE OR OTHER DEALINGS IN THE
%% SOFTWARE.

% The font could be set to Windows-specific Calibri by using the 'calibri' option
\documentclass[]{mcdowellcv}

% For mathematical symbols
\usepackage{amsmath}
\usepackage{hyperref}
\usepackage{endnotes}
\renewcommand{\notesname}{}
\renewcommand{\enoteheading}{}

% Set applicant's personal data for header
\name{Christian Rebischke}
\address{Marie-Hedwig-Straße 13 \linebreak Apartment 321 \linebreak Clausthal-Zellerfeld 38678}
\contacts{+49 151 6190 2666 \linebreak chris@nullday.de \linebreak \url{https://nullday.de}}

\begin{document}

% Print the header
\makeheader

% Print the content
\begin{cvsection}{Employment}
\begin{cvsubsection}{Student Asisstant}{TU Clausthal, Datacenter}{Apr 2016 --- April 2020}

\begin{itemize}
\item Built a proof of concept for deploying Virtual Tunnel End Points (VTEPs) with Ansible on Linux machines for EVPN BGP/VXLAN.
\item Implemented an automated system in Python for fetching IPS firewall alerts via REST API and mailing them to responsible system administrators. This reduced the toil of writing 5--25 mails daily manually.
\item Improved system security and reliability with setting up an OpenVAS vulnerability scanner.
\item Reduced MTTR from one work day to one hour with automating a Freeradius/Radsecproxy/MySQL based AAA infrastructure with Ansible.
\item Showed ownership with maintaning a Proxmox VE cluster consisting of 25 physical nodes.
\item Designed and implemented a command line tool in Python for deploying TLS certificates and private keys on a central firewall for inbound TLS inspection.
\item Evaluated Kubernetes for increasing reliability and introducing micro segmentation via namespace segregation
\item Set up a distributed monitoring system with the help of Traefik, Prometheus and Grafana for monitoring Service Level Indicators (SLIs) for different institutions within the university campus.
\item Gave a talk about Freeradius and Radsecproxy deployment via Ansible on the DFN-BT (annual German research network meetup): \endnote{\url{https://www.dfn.de/fileadmin/3Beratung/Betriebstagungen/bt70/BT70_MobileIT_Konfiguration_FreeRADIUS_und_radsecproxy_mit_Ansible_Strauf_Rebischke.pdf}}
\item Achieved a relation of LDAP users and IP addresses for writing user/IP specific firewall rules via implementing a REST API as middleware between a proprietary service, Freeradius and OpenVPN.
\item Additional key technologies being used: NSCA, NRPE, SNMP, Nginx, Apache, NAPALM, Ansible, NFSv4 over Kerberos, Elasticsearch, Logstash, Kibana, Ansible, REST APIs, Docker.
\end{itemize}
\end{cvsubsection}

\begin{cvsubsection}{Student Assistant}{TU Clausthal Inst.\@ of Software Systems Engineering}{Oct 2016 --- Sep 2017}
\bigskip
\begin{itemize}
\item Build a tool chain for exporting Matlab Simulink models into the Functional Mockup Unit (FMU) format.
\item Developed components for a model transformation tool suite in the project \emph{Spectral Analsysis of Software Architecture}
\item Enhanced code quality with establishing the Continuous Code Quality tool Sonarqube.
\item Key technologies being used: Java, Gradle, Matlab, SVN
\end{itemize}
\end{cvsubsection}

\begin{cvsubsection}{Student Assistant}{TU Clausthal Inst.\@ of Mathematics}{Apr 2014 --- Sep 2017}
\bigskip
\begin{itemize}
\item Increased system reliability with monitoring via the Nagios fork Centreon.
\item Build software packages for Ubuntu (deb) and CentOS (rpm).
\item Has been the system administrator for Linux and Windows machines and gave first level support.
\item Technologies being used: Bash, NFSv4 with Kerberos, Apache, CUPS, MySQL.
\end{itemize}
\end{cvsubsection}

\end{cvsection}

\begin{cvsection}{Education}
\begin{cvsubsection}{B.Sc. Computer Science}{Technical University Clausthal}{Oct 2013 --- May 2019}
\begin{itemize}
	\item Seminar paper: Amazon AWS (EC2 virtual Server and EC2 container) (German) \endnote{\url{https://github.com/shibumi/aws-ec2-project-paper}}
	\item Seminar paper: Openstack (internal structure and overview) (German) \endnote{\url{https://github.com/shibumi/openstack-project-paper}}
	\item Seminar paper: Tor (a short introduction in The Onion Routing) (German) \endnote{\url{https://github.com/shibumi/Tor-project-paper}}
	\item Bachelor thesis: Evaluation of a distributed monitoring system for the TU Clausthal Campus (German) \endnote{\url{https://github.com/shibumi/bachelor-thesis}}\endnote{\url{https://github.com/shibumi/bachelor-kolloquium}}
\end{itemize}
\end{cvsubsection}
\begin{cvsubsection}{M.Sc. Computer Science}{Technical University Clausthal}{Oct 2018 --- Oct 2020}
Current project is finding a theoretical approach for micro service identification and characterization for service matching via the Semantic Web and Ontologies in the research project \textit{Basic technologies and engineering methods for emergent genesis and semantic composition of IoT ecosystems}. The research project will be finished in April 2020.
\end{cvsubsection}

\end{cvsection}
\begin{cvsection}{Open Source Contributions}
\begin{cvsubsection}{Arch Linux}{\url{https://archlinux.org}}{Jan 2015 --- Now}
\begin{itemize}
\item \textbf{Security Advisories} Verifying known Common Vulnerabilities and Exposures (CVEs) in Arch Linux packages.
\item \textbf{Hardening} Improving security of Arch Linux packages and infrastructure.
\item \textbf{Package Maintaner} Building source code into Arch Linux binary packages for distribution, committing patches and supporting the community.
\item \textbf{Release Engineering} Vagrant, qcow2 and Docker image builds for Arch Linux.
\end{itemize}
\end{cvsubsection}

\begin{cvsubsection}{Projects}{\url{https://github.com/shibumi}}{}
\begin{itemize}
\item \textbf{Arch Linux Boxes} Building reliable infrastructure for automated monthly Vagrant and qcow2 image builds with Ansible and Hashicorp Packer. This project includes a small python script that reduces the toil of 1 hour per month to manually check for the monthly needed fresh Arch Linux ISO image. \endnote{\url{https://github.com/archlinux/arch-boxes}}.
\item \textbf{nullday.de} My personal blog with a 100/100 TLS rating \endnote{\url{https://www.ssllabs.com/ssltest/analyze.html?d=nullday.de}}, a 130/100 HTTP headers rating \endnote{\url{https://observatory.mozilla.org/analyze/nullday.de}} and a 100/100 Google PageSpeed Insights rating.
\item \textbf{htpwd} A Go implementation of Apaches \textit{htpasswd}.\endnote{\url{https://github.com/shibumi/htpwd}}
\item \textbf{ryoukai} My i3 statusbar written in Go.\endnote{\url{https://github.com/shibumi/ryoukai}}
\item \textbf{nspawn.org} A hub for systemd-nspawn container images and bootable GPT machine images available on \url{https://nspawn.org}
\item \textbf{SRE-Cheatsheet} I am working on a small Site Reliability Engineering cheat sheet for beginning SREs: \url{https://github.com/shibumi/SRE-cheat-sheet}
\end{itemize}
\end{cvsubsection}
\end{cvsection}

\begin{cvsection}{Languages, Additional Technologies and Interests}
\begin{cvsubsection}{}{}{}
\begin{itemize}
\item \textbf{Natural Languages} German, English
\item \textbf{Programming Languages} Bash, Python, Golang, C, C++, Java, x86 Assembly
\item \textbf{Interests} Site-Reliability Engineering, Devops, Network Infrastructure, Reverse Engineering, Forensics, Penetration Testing, Red Team/Blue Team, Blackbox/Whitebox Testing, Malware Analysis, Server Hardening, Network Security.
\end{itemize}
\end{cvsubsection}
\end{cvsection}

\begin{cvsection}{Footnotes}
\theendnotes
\end{cvsection}

\end{document}
